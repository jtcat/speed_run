%        File: hello.tex
%     Created: Tue Nov 08 02:00 PM 2022 W
% Last Change: Tue Nov 08 02:00 PM 2022 W
%
\documentclass[11pt,a4paper,titlepage]{article}

\usepackage{blindtext}
\usepackage{listings}
\usepackage[margin=1in]{geometry}

\lstset{ 
	backgroundcolor=\color{white},   % choose the background color; you must add \usepackage{color} or \usepackage{xcolor}; should come as last argument
	basicstyle=\small\ttfamily,        % the size of the fonts that are used for the code
	breakatwhitespace=false,         % sets if automatic breaks should only happen at whitespace
	breaklines=true,                 % sets automatic line breaking
	captionpos=b,                    % sets the caption-position to bottom
	commentstyle=\color{mygreen},    % comment style
	deletekeywords={\ldots},            % if you want to delete keywords from the given language
	escapeinside={\%*}{*)},          % if you want to add LaTeX within your code
	extendedchars=true,              % lets you use non-ASCII characters; for 8-bits encodings only, does not work with UTF-8
	frame=lines,	                   % adds a frame around the code
	keepspaces=true,                 % keeps spaces in text, useful for keeping indentation of code (possibly needs columns=flexible)
	keywordstyle=\color{blue},       % keyword style
	language=C,                 % the language of the code
	morekeywords={*,\ldots},            % if you want to add more keywords to the set
	numbers=left,                    % where to put the line-numbers; possible values are (none, left, right)
	numbersep=5pt,                   % how far the line-numbers are from the code
	numbers=none,
	rulecolor=\color{black},         % if not set, the frame-color may be changed on line-breaks within not-black text (e.g. comments (green here))
	showspaces=false,                % show spaces everywhere adding particular underscores; it overrides 'showstringspaces'
	showstringspaces=false,          % underline spaces within strings only
	showtabs=false,                  % show tabs within strings adding particular underscores
	stringstyle=\color{mymauve},     % string literal style
	tabsize=2,	                   % sets default tabsize to 2 spaces
	title=\lstname,                   % show the filename of files included with \lstinputlisting; also try caption instead of title
	literate=
	{á}{{\'a}}1 {é}{{\'e}}1 {í}{{\'i}}1 {ó}{{\'o}}1 {ú}{{\'u}}1
	{Á}{{\'A}}1 {É}{{\'E}}1 {Í}{{\'I}}1 {Ó}{{\'O}}1 {Ú}{{\'U}}1
	{à}{{\`a}}1 {è}{{\`e}}1 {ì}{{\`i}}1 {ò}{{\`o}}1 {ù}{{\`u}}1
	{À}{{\`A}}1 {È}{{\'E}}1 {Ì}{{\`I}}1 {Ò}{{\`O}}1 {Ù}{{\`U}}1
	{ä}{{\"a}}1 {ë}{{\"e}}1 {ï}{{\"i}}1 {ö}{{\"o}}1 {ü}{{\"u}}1
	{Ä}{{\"A}}1 {Ë}{{\"E}}1 {Ï}{{\"I}}1 {Ö}{{\"O}}1 {Ü}{{\"U}}1
	{â}{{\^a}}1 {ê}{{\^e}}1 {î}{{\^i}}1 {ô}{{\^o}}1 {û}{{\^u}}1
	{Â}{{\^A}}1 {Ê}{{\^E}}1 {Î}{{\^I}}1 {Ô}{{\^O}}1 {Û}{{\^U}}1
	{ã}{{\~a}}1 {ẽ}{{\~e}}1 {ĩ}{{\~i}}1 {õ}{{\~o}}1 {ũ}{{\~u}}1
	{Ã}{{\~A}}1 {Ẽ}{{\~E}}1 {Ĩ}{{\~I}}1 {Õ}{{\~O}}1 {Ũ}{{\~U}}1
	{œ}{{\oe}}1 {Œ}{{\OE}}1 {æ}{{\ae}}1 {Æ}{{\AE}}1 {ß}{{\ss}}1
	{ű}{{\H{u}}}1 {Ű}{{\H{U}}}1 {ő}{{\H{o}}}1 {Ő}{{\H{O}}}1
	{ç}{{\c c}}1 {Ç}{{\c C}}1 {ø}{{\o}}1 {å}{{\r a}}1 {Å}{{\r A}}1
	{€}{{\euro}}1 {£}{{\pounds}}1 {«}{{\guillemotleft}}1
	{»}{{\guillemotright}}1 {ñ}{{\~n}}1 {Ñ}{{\~N}}1 {¿}{{?`}}1 {¡}{{!`}}1 
}

\usepackage{color}

\definecolor{mygreen}{rgb}{0,0.6,0}
\definecolor{mygray}{rgb}{0.5,0.5,0.5}
\definecolor{mymauve}{rgb}{0.58,0,0.82}


\newcommand{\extrang}[1]{\textit{#1}}
\newcommand{\srcdir}{..}

\title{
	\textbf{Algoritmos e Estruturas de Dados}

	speed\_run
}

\author{
	João Catarino\\
	NMec: 93096
	\and
	Rúben\\
	NMec: 10000
	\and
	Nuno\\
	NMec: 10000
}

\begin{document}
\setcounter{secnumdepth}{1}
\maketitle
\section{Introdução}
Este trabalho tem como objetivo desenvolver algoritmos que sejam capazes
de encontrar o menor número de passos necessários para resolver o seguinte problema:
\section{Soluções}
A solução original utiliza uma função recursiva para tentar todas as combinações
de velocidade possíveis dentro dos limites de cada casa, guardando sempre a melhor solução encontrada até ao momento.
As soluções seguintes, com exceção da última, partem desta, alterando par
\subsection{Original}
\subsection{Original Improved}
\subsection{Advance and retreat}
Este algoritmo foi feito de raiz. Tem como
princípio tentar a qualquer passo avançar com a velocidade mais alta.
Em cada passo, a escolha de velocidade é representada por um incremento
($-$1, 0 ou 1). O programa começa sempre por tentar o maior incremento.
Para verificar passos possíveis, utilizam-se os seguintes métodos:

Calcular a distância de paragem para cada velocidade possível em cada passo para evitar correr para além do fim do
trajeto.

Verificar se uma ``passada'' quebra os limites de velocidade das casas
pelas quais passaria.

Feito o passo, a escolha de incremento é guardada num \extrang{array}
na posição associada ao número do passo. Desta forma, este \extrang{array}
guarda as escolhas feitas até ao passo atual.
Quando um passo é impossível de executar a qualquer das velocidades possíveis nesse
passo, o algoritmo recua um passo e tenta reduzir o incremento de velocidade até que
consiga avançar novamente. Não sendo possível avançar com nenhum dos incrementos,
o programa recua novamente, e assim sucessivamente.
\section{Código}
\subsection{Advance and retreat}
\lstinputlisting[firstline=69, lastline= 185]{\srcdir/solution\_speed\_run.c}

\end{document}
