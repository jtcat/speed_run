%        File: hello.tex
%     Created: Tue Nov 08 02:00 PM 2022 W
% Last Change: Tue Nov 08 02:00 PM 2022 W
%
\documentclass[portuguese,11pt,a4paper,titlepage]{article}

%\usepackage{titling}
\usepackage{graphicx}
\usepackage{fancyhdr}
\usepackage{babel}
\usepackage{blindtext}
\usepackage{color}
\usepackage{listings}
\usepackage[T1]{fontenc}
\usepackage[margin=3cm]{geometry}

\newcommand{\nextyear}{\advance\year by 1 \the\year\advance\year by -1}

\setlength{\headheight}{14.2pt}
\fancypagestyle{fancy}{
	\fancyhf{}
	\fancyhead[C]{speed\_run}
	\fancyfoot[R]{
		\textsf{\thepage}
	}
	\fancyfoot[L]{
		\textsf{AED - \the\year/\nextyear}
	}
	\fancyfoot[C]{\includegraphics[height=.8cm]{ua.pdf}}
	\renewcommand{\headrulewidth}{0pt}
}
\pagestyle{fancy}

\definecolor{mygreen}{rgb}{0,0.6,0}
\definecolor{mygray}{rgb}{0.5,0.5,0.5}
\definecolor{mymauve}{rgb}{0.58,0,0.82}

\lstset{ 
	backgroundcolor=\color{white},   % choose the background color; you must add \usepackage{color} or \usepackage{xcolor}; should come as last argument
	basicstyle=\small\ttfamily,        % the size of the fonts that are used for the code
	breakatwhitespace=false,         % sets if automatic breaks should only happen at whitespace
	breaklines=true,                 % sets automatic line breaking
	captionpos=b,                    % sets the caption-position to bottom
	commentstyle=\color{mygreen},    % comment style
	deletekeywords={\ldots},            % if you want to delete keywords from the given language
	escapeinside={\%*}{*)},          % if you want to add LaTeX within your code
	extendedchars=true,              % lets you use non-ASCII characters; for 8-bits encodings only, does not work with UTF-8
	frame=lines,	                   % adds a frame around the code
	keepspaces=true,                 % keeps spaces in text, useful for keeping indentation of code (possibly needs columns=flexible)
	keywordstyle=\color{blue},       % keyword style
	language=C,                 % the language of the code
	morekeywords={*,\ldots},            % if you want to add more keywords to the set
	numbers=left,                    % where to put the line-numbers; possible values are (none, left, right)
	numbers=none,
	rulecolor=\color{black},         % if not set, the frame-color may be changed on line-breaks within not-black text (e.g. comments (green here))
	showspaces=false,                % show spaces everywhere adding particular underscores; it overrides 'showstringspaces'
	showstringspaces=false,          % underline spaces within strings only
	showtabs=false,                  % show tabs within strings adding particular underscores
	stringstyle=\color{mymauve},     % string literal style
	tabsize=2,	                   % sets default tabsize to 2 spaces
	title=\lstname,                   % show the filename of files included with \lstinputlisting; also try caption instead of title
}


\newcommand{\extrang}[1]{\textit{#1}}
\newcommand{\srcdir}{..}

\title{\Huge Algoritmos e Estruturas de Dados\vskip .7em
		\bfseries speed\_run\vskip 1.5em
		\includegraphics{ua.pdf}
}
\author{
	João Catarino\\NMec: 93096\and Rúben\\NMec: 10000\and Nuno\\NMec: 10000
}
\date{novembro de 2022}

\begin{document}
%\begin{titlingpage}
\maketitle
%\end{titlingpage}
\setcounter{secnumdepth}{1}
\section{Introdução}
Este trabalho tem como objetivo desenvolver algoritmos que sejam capazes
de encontrar o menor número de passos necessários para resolver o seguinte problema:
\section{Soluções}
A solução original utiliza uma função recursiva para tentar todas as combinações
de velocidade possíveis dentro dos limites de cada casa, guardando sempre a melhor solução encontrada até ao momento.
As soluções seguintes, com exceção da última, partem desta, alterando par
\subsection{Original}
\subsection{Original Improved}
\subsection{Advance and retreat}
Este algoritmo foi feito de raiz. Tem como
princípio tentar a qualquer passo avançar com a velocidade mais alta.
Em cada passo, a escolha de velocidade é representada por um incremento
($-$1, 0 ou 1). O programa começa sempre por tentar o maior incremento.
Para verificar passos possíveis, utilizam-se os seguintes métodos:

Calcular a distância de paragem para cada velocidade possível em cada passo para evitar correr para além do fim do
trajeto.

Verificar, a partir do incremento mais alto, se uma ``passada'' não  quebra os limites de velocidade das casas pelas quais passaria.

Feito o passo, a escolha de incremento é guardada num \extrang{array}
na posição associada ao número do passo. Desta forma, este \extrang{array}
guarda as escolhas feitas até ao passo atual.
Quando um passo é impossível de executar a qualquer das velocidades possíveis nesse
passo, o algoritmo recua um passo e tenta reduzir o incremento de velocidade até que
consiga avançar novamente. Não sendo possível avançar com nenhum dos incrementos,
o programa recua novamente, e assim sucessivamente.
\vfill
\section{Código}
\subsection{Advance and retreat}
\lstinputlisting[firstline=69, lastline= 185]{\srcdir/solution\_speed\_run.c}

\end{document}
